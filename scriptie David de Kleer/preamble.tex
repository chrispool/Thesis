%%%% PREAMBLE

\documentclass[a4paper,10pt,titlepage]{article}

\usepackage[a4paper]{geometry}

\usepackage{listings}
% listing (voor programmeercode, in dit geval Python)
\lstset{language=Python, basicstyle=\small\fontfamily{pxtt}\selectfont, keywordstyle=\bfseries, breaklines=true, morekeywords={then},
	showstringspaces=false, aboveskip=10pt, belowskip=10pt, columns=fullflexible}
% dotted table of contents
\usepackage{tocloft}
\renewcommand{\cftsecleader}{\cftdotfill{\cftdotsep}}
% meer ruimte in table of contents
\setlength{\cftbeforesecskip}{10pt}
\setlength{\cftbeforesubsecskip}{10pt}
\setlength{\cftbeforesubsubsecskip}{2pt}
\usepackage[utf8]{inputenc}
\usepackage[T1]{fontenc}
\usepackage{apacite}
\usepackage{upquote} % voor enkele quotes in listings
\usepackage[dutch]{babel}
\usepackage{graphicx}
\usepackage[sort]{natbib}
\usepackage{eurosym} % euro symbolen
\usepackage{framed}
\usepackage[hyphens]{url}
\usepackage{verse} % voor gedichten
% Plaats footnotes helemaal onderaan de pagina, niet alleen onderaan de tekst 
% (waardoor voetnoten tussen de tekst en een figuur terecht kunnen komen).
\usepackage[bottom,hang,flushmargin]{footmisc}
\usepackage{wallpaper}
\usepackage[font={it}]{caption} % onderschriften in italics
\usepackage{float}
\usepackage[compact]{titlesec}
\usepackage{enumitem}
% hvfloat voor horizontale tabellen
\usepackage{hvfloat}
\usepackage[hypcap]{caption}
\usepackage[hidelinks]{hyperref}
\setlength\parindent{0pt} % gebruik nooit indents
% titlespacing* zorgt ervoor dat er geen indents worden geplaatst na het begin 
% van een sectie
\titlespacing{\section}{0pt}{10pt}{10pt}
\titlespacing{\subsection}{0pt}{10pt}{10pt}
\titlespacing{\subsubsection}{0pt}{10pt}{10pt}
\titlespacing{\paragraph}{0pt}{10pt}{10pt}
% marges na figuren
\setlength{\belowcaptionskip}{0pt}

\addtolength{\topmargin}{-.5in}
\addtolength{\textheight}{1.06in}
\setcounter{secnumdepth}{5}                   % voor subsubsubsections: gebruik 
                                              % paragraph
\setcounter{tocdepth}{5}                      % blijf de inhoud van de PDF 
                                              % juist weergeven
                                              
\def\wl{\par \vspace{\baselineskip}\noindent} % white space tussen alinea's: 
                                              % gebruik \wl
\def\vl{\\[9pt]}                              % 9pt verticale ruimte, 
                                              % alternatief voor \wl
\def\s{\section}                              % \s voor een section
\def\ss{\subsection}                          % \ss voor een subsection
\def\sss{\subsubsection}                      % \sss voor een subsubsection
\def\p{\paragraph}                            % \p voor een paragraph 
                                              % (of subsubsubsection)
\def\ttt{\texttt}                             % teletype text

\newenvironment{bullets}                      % bullets in plaats van itemize
{\begin{itemize}}                             % (ik vergeet itemize altijd xD)
{\end{itemize}}

\def\image{\includegraphics}                  % image makkelijker te onthouden
\def\it{\textit}                              % kortere italics
\def\bf{\textbf}                              % bold, korter

% Voorbeeld voor een plaatje met label

%\begin{figure}[H] % H betekent: voeg HIER in, op deze plek
%  \centering
%    \image[width=0.5\textwidth]{zeemeeuw}
%    \caption{Close-up van een meeuw}
%  \label{meeuw}
%\end{figure}
 
%Figuur \ref{meeuw} toont een foto van een meeuw.

% CTRL+D: comment
% CTRL+SHIFT+D: uncomment

% Monospace schrijven: texttt{Monospace!}

% Underscore typen? Gebruik \_.

% Geen ruimte tussen bullets/enumerates? Gebruik \itemsep0em.
